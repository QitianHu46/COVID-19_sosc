\documentclass{article}
\usepackage[utf8]{inputenc}
\usepackage[english]{babel}
\usepackage[]{amsthm} %lets us use \begin{proof}
\usepackage[]{amssymb} %gives us the character \varnothing
\usepackage{amsmath}
\usepackage[shortlabels]{enumitem}


\title{COVID-19 Relief Fellowship Application}
\author{Qitian (Jason) Hu}

%This information doesn't actually show up on your document unless you use the maketitle command below

\linespread{1.1}
\usepackage[margin=1.2in]{geometry}
\usepackage{graphicx}
\begin{document}

\maketitle %This command prints the title based on information entered above
\textit{1. Briefly describe your project and what area of COVID will your solution focus on (500 words max).}\\
This research project aims to explore how and why people respond differently to policies targeting COVID-19 in different regions. I observe that in some countries, like China, although stringent policies like lockdown of large cities are applied, there are still widespread supports. While in some states in the U.S., there are widespread protests against governments shutting down public facilities. People in different regions respond very differently when similar policies are made to similar external pressure. How do these different responses relate to socio-economic conditions (GDP, income, living standard), racial composition, the seriousness of the pandemic in the area, and the form and structure of the government? I will systematically measure the relationship between public opinion and policies with big data from social media and open datasets. Relations between these variables will be formalized with econometric methods. The project aims to provide a systematic framework with which we can understand how are people's opinions of the pandemic and policies are shaped by their unique environment, and this enables policymakers to better understand their policies and balance between the effectiveness of policies and potential public response. \\

\textit{2. Who will this solution benefit? Is there a specific target group your solution will impact?}\\
Policymakers will be a group of people that directly benefit from this research. Although many policies are proven effective to tackle COVID-19 somewhere, they may not work everywhere due to the unique socio-economic condition and culture in each region. This research will enable politicians to prevent policies that cause serious public objections and thus fail to work in the future policy-making processes. Other potential beneficiaries include people who would benefit from the improved policy-making process, and I also hope this work can contribute to the ongoing discussion of the socio-economic impact of the virus in academia. \\

\textit{3. Why are you interested in this project?} \\
Being a Chinese living and studying in the US, I strongly feel this huge discrepancy in the people's response to COVID-19 and relevant government policies. I think simply attributing this difference to culture is far from satisfactory, and I want to rigorously explore this topic with insights from social science. In addition, this pandemic, although cruel and cunning, asserts a similar pressure to every country in the world, thus providing a good natural experiment. Social scientists can get a deeper understanding of the uniqueness and connection between society and government in these countries by observing different countries' responses to this uniform stimulus. \\

\textit{4. Your independent work this summer will culminate in a competition where a winning solution will be selected and granted \$5,000 for implementation. Briefly describe what project implementation might look like and include a rough timeline for implementation.}\\
The fund will be used to further, publicize, and apply the results in various ways. Firstly, within a month since the project finished, I will use the fund to build a website to publicize relevant findings and methodologies. Meanwhile, I will introduce my findings to media, think tanks, and interested policymakers and try to incorporate this framework of analysis into real policy-making processes. If time and resource permit, I would like to generalize these findings by establishing a website on which policymakers, scholars, and the general public can have data-oriented analysis to major public health issues and the policy and public opinion in response to them.\\

\textit{5. What challenges with implementation do you anticipate?}\\
I think the most important challenge is the depth and validity of the analysis. This is an extremely complicated issue and it is difficult to provide a convincing framework and data to explain this issue. However, with a plan to scrape, analyze the data and guidance from faculty members, I have the confidence to breakdown these problems. Secondly, the project might not be able to be successfully translated to policies, either as policy-makers regard this analysis as irrelevant, or the COVID-19 crisis has already passed by the time my research is completed. If the analysis has sufficient rigor and explanatory power, I believe it will be useful to some people, either policymakers, think tank analysts, or other social researchers. For the issue that the analysis may no longer be relevant, I think the issues exposed by COVID-19 does not only apply to this specific event, but has important implication for every public crisis that requires a government response.\\



\textit{6. How you would define a/o measure success for this project?}\\
First of all, I believe a successful analysis should be consistent with existing evidence and data that are relevant to people's response to COVID-19 policies, and powerful research should be able to be publicized on peer-reviewed platforms. In addition to the explanatory power aspect, I will define the project as successful if it helps people, either it helps policymakers make better decisions, or help people better understand current policies and responses would be great. Hopefully, my work can also contribute to the literature that explores human societies in crisis. 

\textit{7. Do you have a mentor for this project? If so, please list their name and contact information and how they will support you. If you do not have a mentor, we can assign you one.}
} \\
Yes. My mentor is Professor Christina Jenq from New York University, Shanghai. Her email is: cj385@nyu.edu.





\end{document}